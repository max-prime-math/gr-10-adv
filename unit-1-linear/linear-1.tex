\documentclass[12pt,fleqn]{book}

\usepackage{titlesec}
\usepackage{caption}

\titleformat{\chapter}[display]
  {\normalfont\huge\bfseries}{}{0pt}{\Huge}
\titlespacing{\chapter}
  {0pt}{10pt}{40pt}

	\newenvironment{amatrix}[1]{%
	  \left[\begin{array}{@{}*{#1}{r}|r@{}}
	}{
	  \end{array}\right]
	}


\usepackage{amsmath,amssymb,amsfonts,graphicx,tasks,tikz,pgfplots}
\usetikzlibrary{arrows}
\pgfplotsset{compat=1.17}

\usepackage[a4paper,margin=0.5in,footskip=.5cm]{geometry}

\setlength\parindent{0pt}

\usepackage{amsmath,amssymb,amsfonts,graphicx,tasks,tikz,pgfplots}
\usepackage{amsthm,thmtools}
\usetikzlibrary{arrows,quotes}

\usepackage{array}
\newcommand{\PreserveBackslash}[1]{\let\temp=\\#1\let\\=\temp}
\newcolumntype{C}[1]{>{\PreserveBackslash\centering}p{#1}}
\newcolumntype{R}[1]{>{\PreserveBackslash\raggedleft}p{#1}}
\newcolumntype{L}[1]{>{\PreserveBackslash\raggedright}p{#1}}

% Two-branch graph with equal scale and only max tick labels
% #1 = y1(x), #2 = y2(x), #3 = xmin, #4 = xmax, #5 = ymin, #6 = ymax,
% #7 = tick distance (ignored for equal scale), #8 = extra axis opts, #9 = plot opts
\newcommand{\graphpair}[9]{%
\begin{tikzpicture}[baseline=(current bounding box.north)]
  \begin{axis}[
    axis lines=middle,
    axis line style={very thick},
    grid style={thin,densely dotted,black!50},
    grid=major,
    xmin=#3-.2, xmax=#4+.2,
    ymin=#5-.2, ymax=#6+.2,
    xlabel=$x$, ylabel=$y$,
    axis equal image,
    % Only show largest tick values on each axis
    xticklabels={},
	yticklabels={},
	xtick distance=1, ytick distance=1,
	extra x ticks={#4},
	extra y ticks={#6},
    enlargelimits=false,
    % Allow optional extra axis settings
    #8
  ]
    \addplot[domain=#3:#4, samples=500, #9] (x,{#1});
    \addplot[domain=#3:#4, samples=500, #9] (x,{#2});
  \end{axis}
\end{tikzpicture}%
}


\newcommand{\graph}[8]{
\begin{tikzpicture}[baseline=(current bounding box.north)]
  \begin{axis}[
    axis lines=middle,
    axis line style={very thick},
    grid style={thin,densely dotted,black!50},
    grid=major,
      xtick distance=#6, ytick distance=#6,
      xmin=#2-.2, xmax=#3+.2,
      ymin=#4-.2, ymax=#5+.2,
      xlabel=$x$,
      ylabel=$y$,
    xticklabels={},
	yticklabels={},
	xtick distance=1, ytick distance=1,
	extra x ticks={#3},
	extra y ticks={#5},
      #7]
      \addplot[
      domain=#2:#3,
      samples=500,#8]
      (x,{#1});
    \end{axis}
  \end{tikzpicture}
}

\newcommand{\graphtwo}[9]{
\begin{tikzpicture}[baseline=(current bounding box.north)]
  \begin{axis}[
    axis lines=middle,
    axis line style={very thick},
    grid style={thin,densely dotted,black!50},
    grid=major,
      xtick distance=#7, ytick distance=#7,
\usepackage{amsmath,amssymb,amsfonts,amsthm,thmtools}
      xmin=#3-.2, xmax=#4+.2,
      ymin=#5-.2, ymax=#6+.2,
      xlabel=$x$,
      ylabel=$y$,
      #8]
      \addplot[
      domain=#3:#4,
      samples=500,#9]
      (x,{#1});
      \addplot[
      domain=#3:#4,
      samples=500,#9]
      (x,{#2});
    \end{axis}
  \end{tikzpicture}
}

\newcommand{\curve}[7]{% function, xmin, xmax, ymin, ymax, 
\begin{tikzpicture}[baseline=(current bounding box.north)]
  \begin{axis}[
    axis lines=middle,
    axis line style={very thick},
    grid style={thin,densely dotted,black!50},
    grid=major,
    xticklabels={},
	yticklabels={},
	xtick distance=1, ytick distance=1,
	extra x ticks={#3},
	extra y ticks={#5},
      xmin=#2-.2, xmax=#3+.2,
      ymin=#4-.2, ymax=#5+.2,
      xlabel=$x$,
      ylabel=$y$,
      #6]
      \addplot[
      domain=#2:#3,
      samples=500,#7]
      (x,{#1});
    \end{axis}
  \end{tikzpicture}
}

\newcommand{\blankgraph}[6]{
\begin{tikzpicture}[baseline=(current bounding box.north)]
  \begin{axis}[
    width=#5in,height=#6in,
    axis lines=middle,
    axis line style={very thick},
    grid style={thin,densely dotted,black!50},
    grid=major,
    xticklabels={},
	yticklabels={},
	xtick distance=1, ytick distance=1,
	extra x ticks={#2},
	extra y ticks={#4},
      xtick distance=1, ytick distance=1,
      xticklabels={1},
      yticklabels={1},
      xmin=#1-.2, xmax=#2+.2,
      ymin=#3-.2, ymax=#4+.2,
      xlabel=$x$,
      ylabel=$y$]
    \end{axis}
  \end{tikzpicture}
}

\newcommand{\blankaxes}[2]{
\begin{tikzpicture}[baseline=(current bounding box.north)]
  \begin{axis}[
    width=#1in,height=#2in,
    axis lines=middle,
    axis line style={very thick},
    grid=major,
    xtick distance=2, ytick distance=2,
    xmin=-1, xmax=1,
    ymin=-1, ymax=1,
    xlabel=$x$,
    ylabel=$y$]
  \end{axis}
\end{tikzpicture}
}

\newcommand{\ds}{\displaystyle}

\newcommand{\lr}[1]{\left(#1\right)}

% \usepackage{amsthm}
% \theoremstyle{definition}
% \newtheorem{problem}{}
% \counterwithin*{problem}{section}
% \renewcommand{\theproblem}{\thesubsection.\arabic{problem}}
%
% \newtheorem*{solutionx}{\thesolutionnumber}
% \ExplSyntaxOn
% \tl_new:N \g_gargantuar_solution_tl
%
% \NewDocumentEnvironment{solution}{+b}
%  {
%   \tl_gput_right:Nx \g_gargantuar_solution_tl
%    {
%     \printsolution{\theproblem}{ \exp_not:n { #1 } }
%    }
%  }
%  {}
% \NewDocumentCommand{\printsolutions}{}
%  {
%   \tl_use:N \g_gargantuar_solution_tl
%   \tl_gclear:N \g_gargantuar_solution_tl
%  }
% \NewDocumentCommand{\printsolution}{m +m}
%  {
%   \cs_set:Npn \thesolutionnumber { #1 }
%   \begin{solutionx} #2 \end{solutionx}
%  }
% \ExplSyntaxOff
% \makeatother
%
% \newcommand{\prb}[1]{\begin{problem}#1\end{problem}}
% \newcommand{\sol}[1]{\begin{solution}#1\end{solution}}
% =============================
% TCOLORBOX CONFIGURATION
% =============================

\usepackage{tcolorbox}

\definecolor{black}{rgb}{0,0,0}
\definecolor{dark-gray}{rgb}{.15,.15,.15}
\definecolor{light-gray}{rgb}{.85,.85,.85}
\definecolor{white}{rgb}{1,1,1}

\makeatletter
\newcommand*\ifcounter[1]{%
	\ifcsname c@#1\endcsname
		\expandafter\@firstoftwo
	\else
		\expandafter\@secondoftwo
	\fi
}
\makeatother

\tcbuselibrary{theorems}

\ifcounter{unit}{
	\newtcbtheorem{defn}{Definition \theunit.\hspace{-.3em}}{
		colback=light-gray,colframe=dark-gray,coltitle=white,coltext=black,fonttitle=\bfseries,before skip=20pt plus 2pt,after skip=20pt plus 2pt}{th}
}{
	\newtcbtheorem{defn}{Definition}{
		colback=light-gray,colframe=dark-gray,coltitle=white,coltext=black,fonttitle=\bfseries,before skip=20pt plus 2pt,after skip=20pt plus 2pt}{th}
}

\ifcounter{unit}{
	\newtcbtheorem{thm}{Theorem \theunit.\hspace{-.3em}}{
		colback=light-gray,colframe=dark-gray,coltitle=white,coltext=black,fonttitle=\bfseries,before skip=20pt plus 2pt,after skip=20pt plus 2pt}{th}
}{
	\newtcbtheorem{thm}{Theorem}{
		colback=light-gray,colframe=dark-gray,coltitle=white,coltext=black,fonttitle=\bfseries,before skip=20pt plus 2pt,after skip=20pt plus 2pt}{th}
}



\pagestyle{plain}
\usepackage[lastexercise,answerdelayed]{exercise}
\usepackage{totcount}
\regtotcounter{Exercise} % register the counter for getting the total
\usepackage{chngcntr}
\counterwithin{Exercise}{chapter}
\counterwithin{Answer}{chapter}
\usepackage{xassoccnt}
\NewTotalDocumentCounter{totalex}
\usepackage{titleref}
\usepackage{etoolbox}

\renewcommand{\ExerciseHeader}{ \stepcounter{totalex}\ifnumcomp{\value{Exercise}}{=}{1}{\ifnumcomp{\thechapter}{=}{1}{}{\vspace{10pt}}\noindent\Large\textbf{Exercises}\par\vspace{10pt}}{}\noindent\normalsize\bfseries\ExerciseHeaderNB\ExerciseHeaderDifficulty.}
\renewcommand{\AnswerHeader}{\ifnumcomp{\value{Exercise}}{=}{1}{\ifnumcomp{\thechapter}{=}{1}{}{\vspace{10pt}}\noindent\Large\textbf{Section\ \thechapter\ Solutions}\par\vspace{10pt}}{}\noindent\normalsize\bfseries\ExerciseHeaderNB.\ }

\usepackage{hyperref}

\newcommand{\prb}[1]{\begin{Exercise}#1\end{Exercise}}
\newcommand{\sol}[1]{\begin{Answer}#1\end{Answer}}

\begin{document}
\noindent
\thispagestyle{empty}
Grade 10 Advanced Math \hfill Name: \hspace{2in}
\medskip\hrule
\noindent

\vfill

\begin{center}
	{\bf \huge Chapter 1: Linear Functions \\ \& Matrices}
\end{center}

\vfill
\vfill

\clearpage

\setcounter{page}{1}

{\bf \huge Introduction }
\\[2em]
A linear relation is a relationship between two or more variables for which one variable may be obtained from the other(s) using only multiplication and addition of Real Numbers.
\\[1em]
E.g. Explain why the following are linear relations:
\begin{enumerate}
	\item $2x+3y = 6$
	\vspace{1in}
	\item $3a-2b = 5c$
	\vspace{1in}
\end{enumerate}
Some relations are not linear.  We call these non-linear.  Some sure-fire ways of determining if a relation is non-linear is by looking for the following operations/functions on the variables:
\begin{itemize}
	\item exponents,
	\item trigonometric functions,
	\item square roots,
	\item division.
\end{itemize}
Keep in mind that we are only concerned if those operations are done to the variables.
\\[1em]
E.g. Explain why $y=\sqrt{2x}$ is non-linear but $y=\sqrt 2 x$ is linear.
\\[1in]
Write a few examples of relations that have exponents, trigonometry, and square roots in them, but are still linear.

\chapter{1. Graphing Linear Functions.}
Linear functions get their name from what they look like on a coordinate plane.  No matter how many variables you use, when a linear relation is graphed the result is a straight line.
\\[1em]
On a two dimensional Cartesian coordinate plane, there are two main methods of graphing a linear relation.
\begin{itemize}
	\item Slope-intercept form $y=mx+b$.
	\item Slope-point form $y=m(x-x_0)+y_0$.
\end{itemize}
Here, $m$ represents slope, $b$ represents the $y$-intercept, and $(x_0,y_0)$ is any point on the line.
\\[1em]
The graph below could be expressed in several different ways.  Using slope-intercept form, we can express $y$ as a function of $x$ by writing $\ds y=\frac 23 x +2$, since the slope $m$ is equal to $\frac 23$ (for every 2 units it moves up, it moves 3 units to the right).
\\[1em]
\graph{(2/3)*x+1}{-4}{4}{-4}{4}{1}{}{<->}
\\[1em]
Another way we can express the function above is by using slope-point form.  Keep in mind, when we use this form we can use any point on the line that we please.  Since the function passes through $(-3,-1)$ we can write it as
\begin{align*}
    y&=\frac 23 (x-(-3))+(-1)\\
    &=\frac 23 (x+3)-1.
\end{align*}

\clearpage
\prb{The formula for the slope-point form of a line is:}\vspace{4em}
\sol{$y=m(x-x_0)+y_0$; $m$ is the slope, $(x_0,y_0)$ is any point on the line.}
\prb{The formula for slope-intercept form of a line is: }\vspace{4em}
\sol{$y=mx+b$; $m$ is the slope, $(0,b)$ is the $y$-intercept.}
\prb{The slope of the line $Ax+By=C$ is:\\[4em]}
\sol{$m=-\frac AB$.}
\prb{Find the slope and $y$-intercept of each function.
\begin{tasks}(2)
\task $3x-2y=6$		\\[4em]
\task $4x+3y=12$	\\[4em]
\task $2x-5y=-7$ 	\\[4em]
\task $5x+2y=0$ 	\\[4em]
\task $x-4y=-4$ 	\\[4em]
\task $6x-y=-3$ 	\\[4em]
\end{tasks}
}
\sol{\
\begin{tasks}(2)
\task Slope: $\frac 32$, $y$-intercept: $(0,-3)$
\task Slope: $-\frac 43$, $y$-intercept: $(0,4)$
\task Slope: $\frac 25$, $y$-intercept: $(0,\frac 75)$
\task Slope: $-\frac 52$, $y$-intercept: $(0,0)$
\task Slope: $\frac 14$, $y$-intercept: $(0,1)$
\task Slope: $6$, $y$-intercept: $(0,3)$
\end{tasks}
}
\clearpage
\prb{Graph each linear equation.
\begin{tasks}(2)
\task $\ds 4x-3y=12$\\
  \blankgraph{-10}{10}{-10}{10}{3}{3}
	\vspace{2.3in}
\task $\ds y=-\frac 23 x + 4$\\
  \blankgraph{-10}{10}{-10}{10}{3}{3}
	\vspace{2.3in}
\task $\ds y-3 = \frac 12 (x+4)$\\
  \blankgraph{-10}{10}{-10}{10}{3}{3}
	\vspace{2.3in}
\task $\ds 2x+3y=10$\\
  \blankgraph{-10}{10}{-10}{10}{3}{3}
	\vspace{2.3in}
	\clearpage 

	
\task $\ds y+2 = -\frac 23 (x+5)$\\
  \blankgraph{-10}{10}{-10}{10}{3}{3}
	\vspace{2.3in}
\task $\ds 5x-2y=0$\\
  \blankgraph{-10}{10}{-10}{10}{3}{3}
	\vspace{2.3in}
\task $\ds y-\frac 52 = -\frac 12 \left(x+\frac 32\right)$\\
  \blankgraph{-10}{10}{-10}{10}{3}{3}
	\vspace{2.3in}
\task $\ds y=\frac 53x - \frac 72$\\
  \blankgraph{-10}{10}{-10}{10}{3}{3}
	\vspace{2.3in}
\end{tasks}
}
\sol{\
\begin{tasks}(2)
\task
\graph{(4/3)*x-4}{-10}{10}{-10}{10}{5}{}{}
\task
\graph{-(2/3)*x+4}{-10}{10}{-10}{10}{5}{}{}
\task
\graph{(1/2)*(x+4)+3}{-10}{10}{-10}{10}{5}{}{}
\task
\graph{-(2/3)*x+(10/3)}{-10}{10}{-10}{10}{5}{}{}
\task
\graph{-(2/3)*(x+5)-2}{-10}{10}{-10}{10}{5}{}{}
\task
\graph{(5/2)*x}{-10}{10}{-10}{10}{5}{}{}
\task
\graph{-(1/2)*(x+3/2)+5/2}{-10}{10}{-10}{10}{5}{}{}
\task
\graph{(5/3)*x-7/2}{-10}{10}{-10}{10}{5}{}{}
\end{tasks}
}

\clearpage

\prb{Determine the equation of the line that passes through each pair of points.
\begin{tasks}(2)
\task $(-4, 1)$ and $(6, 1).$
  \vspace{4em}
\task $(-2,-1)$ and $(5,-1).$
  \vspace{4em}
\task $( 1, 4)$ and $(6,-3).$
  \vspace{4em}
\task $(-5,-3)$ and $(6, 2).$
  \vspace{4em}
\task $( a, b)$ and $(c, d).$
  \vspace{4em}
\task $( 0, s)$ and $(t, 0).$
  \vspace{4em}
\end{tasks}
}

\sol{\
\begin{tasks}(2)
\task $\ds y=1$
\task $\ds y=-1$
\task $\ds y=\frac{-7}{5}(x-1)+4$
\task $\ds y=\frac{5}{11}(x+5)-3$
\task $\ds y=\frac{b-d}{a-c}(x-a)+b$
\task $\ds y=\frac{-s}{t}x+s$
\end{tasks}
}

\prb{
  If a line has slope $\ds \frac ab$, what is the slope of a line which is perpendicular to it?
  \vspace{4em}
}
\sol{$\ds -\frac ba$}
\prb{
  If a line is horizontal, what is the slope of any line that is perpendicular to it?
  \vspace{4em}
}
\sol{
  Undefined.
}
\prb{ Find the equation of a line parallel to $x-3y=8$ with the same $y$-intercept as $3x+2y=6$.}
\vspace{4em}
\sol{ $\ds y=\frac{1}{3}x+3$. }

\prb{
  What is the equation of a line that passes through $(a,b)$ and has slope equal to zero?
  \vspace{4em}
}
\sol{$y=b$}
\prb{
  What are the $x$- and $y$-intercept of $ax+by=c$?
  \vspace{4em}
}
\sol{$x$-intercept: $\ds\frac ca$, $y$-intercept: $\ds\frac cb$.}

\prb{
	Since the temperature $0^\circ$C is equal to $32^\circ$F, and $-40^\circ$C is equal to $-40^\circ$F, find a formula that expresses Fahrenheit in terms of Celcius.
	\vspace{4em}
}
\sol{$F=\frac{9}{5}C+32$}

\clearpage

\chapter{Systems of Linear Equations.}
A system of equations is a collection of equations that (usually) share the same variables.  The following is a system:
\begin{align*}
    2x+3y&=5\\
    3x-2y&=2.
\end{align*}
A solution to a system of equations is a value given to each variable that makes each equation in the system true.  The system above has solution $x=1$ and $y=1$.  Another way to write this is $(1,1)$, since the two lines intersect at that point.
\\[1em]
Graph the two lines and verify that the solution given is where the two lines intersect.
\\[1em]
\blankgraph{-1}{4}{-2}{3}{3.5}{3.5}
\\[1em]
Are there any systems that have no solutions?
\\[5em]
Are there any systems that have more than one solution?
\\[5em]
There are many ways of solving a system of equations.  We will discuss three ways in this section.
\begin{enumerate}
  \item Graphing: if you graph all equations on the same coordinate plane, you can visually identify where all of the equations meet.  A solution is any point where each equation passes through.
  \item Substitution: if you can isolate one variable in one of the equations, you can substitute that variable into the other equation.  From this point you have one equation with one variable, and you can use any technique to solve for that variable.  Once you know the value of one variable, you can substitute that back in to either of the original equations and solve for the other variable.
  \item Elimination: if you add or subtract one equation from the other, you can often eliminate a variable.  Once you eliminate one variable, you can now solve for the other variable.  Just like with substitution, once you know the value of one variable, you can use it to find the other.
\end{enumerate}
Use substitution to solve the above system.\\[1in]
Use elimination to solve the above system.\\[1in]
If each equation in our system of equations is a linear equation, we call that system a linear system, or a system of linear equations.  In any linear system, there are three possible outcomes for a solution:
\begin{enumerate}
  \item Exactly one solution: each line passes through the same point.
  \item No solution: lines are parallel and share no common points.
  \item Infinitely many solutions: lines overlap.
\end{enumerate}
If a system has at least one solution we call that system ``consistent,'' and if it has no solutions we call that system ``inconsistent.''
\prb{Determine if the ordered pair is a solution to the linear system.
\begin{tasks}(2)
\task $(5,2)$
\begin{align*}
3x+y&=17\\
2x+3y&=17
\end{align*}
\vspace{1in}
\task $(-3,7)$
\begin{align*}
-6x+6y&=60\\
14y&=68
\end{align*}
\vspace{1in}
\task $(-3,-6)$
\begin{align*}
11x+3y&=-51\\
9x+15y&=-117
\end{align*}
\vspace{1in}
\task $(8,-8)$
\begin{align*}
-7y&=56\\
x+14y&=-103
\end{align*}
\end{tasks}
\vspace{1in}
}
\sol{\
  \begin{tasks}(2)
  \task No.
  \task Yes.
  \task Yes.
  \task No.
  \end{tasks}
}
\prb{Solve each system by graphing.
\begin{tasks}(2)
\task \vspace{-2.5em}
\begin{align*}
2x-y&=3 \\
x+y&=3
\end{align*}
\blankgraph{-10}{10}{-10}{10}{3.5}{3.5}
\vspace{.5in}

\task \vspace{-2.5em}
\begin{align*}
x+2y&=-4 \\
y&=-\frac 12 x+1
\end{align*}
\blankgraph{-10}{10}{-10}{10}{3.5}{3.5}
\vspace{.5in}


\task \vspace{-2.5em}
\begin{align*}
x&=4 \\
3x-2y&=6
\end{align*}
\blankgraph{-10}{10}{-10}{10}{3.5}{3.5}
\vspace{.5in}


\task \vspace{-2.5em}
\begin{align*}
x+y&=-5 \\
-2x+1&=1
\end{align*}
\blankgraph{-10}{10}{-10}{10}{3.5}{3.5}
\vspace{.5in}

\end{tasks}
}
\sol{
\begin{tasks}(2)
\task
x=2, y=1 \\
\graphtwo{2*x-3}{3-x}{-5}{5}{-5}{5}{1}{}{}
\task
No Solutions \\
\graphtwo{0}{0}{-5}{5}{-5}{5}{1}{}{}
\task
x=4, y=3 \\
\graphtwo{100*(x-4)}{3/2*x-3}{-5}{5}{-5}{5}{1}{}{}
\task
x=0, y=-5 \\
\graphtwo{-5-x}{100*x-5}{-5}{5}{-10}{0}{1}{}{}
\end{tasks}
}
\clearpage
\prb{
  Is it possible for a system of linear equations to have exactly two solutions?  Explain.
  \vspace{2in}
}
\sol{No.  A system of linear equations can have either one, zero, or infinitely many solutions.}
\prb{Solve using elimination.
\begin{tasks}(2)
\task \vspace{-2.5em}
\begin{align*}
x+y   &=-5 \\
-2x+1 &=1
\end{align*}

\vspace{2in}

\task \vspace{-2.5em}
\begin{align*}
5x+2y  &= 8 \\
3x+5y  &= 20
\end{align*}

\vspace{2in}

\task \vspace{-2.5em}
\begin{align*}
3x-y  &= 4 \\
x + 2y &=2
\end{align*}

\vspace{2in}

\task \vspace{-2.5em}
\begin{align*}
2x-5y &= 6 \\
3x+2y &= 5
\end{align*}

\vspace{2in}

\task \vspace{-2.5em}
\begin{align*}
3x-2y&=-5 \\
3x+y &=11
\end{align*}

\vspace{2in}

\task \vspace{-2.5em}
\begin{align*}
\frac x2 - \frac {2y}{3} &= 2 \\
\frac x4 + 3y &= -4
\end{align*}

\vspace{2in}

\end{tasks}
}
\sol{
\begin{tasks}(3)
    \task $x=0, y=-5$
    \task $x=0, y=4$
    \task $x=10/7, y=2/7$
    \task $x=37/19, y=-8/19$
    \task $x=17/9, y=16/3$
    \task $x=2, y=-3/2$
\end{tasks}
}

\prb{ If the system of equations has one solution, what is the value of $k$?
\begin{align*}
y &= 3x+2 \\
y &= kx+2
\end{align*}
}
\vspace{1in}
\sol{$k\ne 3$}
\prb{ If the system of equations has no solutions, what is the value of $k$?
\begin{align*}
3x-y &= 6 \\
-6x+2y &= k
\end{align*}
}
\vspace{1in}
\sol{$k\ne -12$}
\prb{Find the value of $k$ for which the system below has infinitely many solutions.
\begin{align*}
3x-y &= 4 \\
-6x+2y &= k
\end{align*}
}
\sol{$k=-8$}
\vspace{1in}
\prb{Solve the system in terms of $a$ and $b$.
\begin{tasks}(2)
\task 	\vspace{-2.4em}
\begin{align*}
x+ay &= b \\
x-ay &= 2b
\end{align*}
\task	\vspace{-2.4em}
\begin{align*}
3ax-y&=b\\
2ax+y&=4b
\end{align*}
\vspace{1in}
\task	\vspace{-2.4em}
\begin{align*}
5x+2y &= a \\
x-y &= b
\end{align*}
\task 	\vspace{-2.4em}
\begin{align*}
	ax+by &= -26 \\
bx+ay &= 7
\end{align*}
\end{tasks}
}
\sol{
\begin{tasks}(2)
    \task $x=\frac{3b}{2}, y=\frac{-b}{2a}$
    \task $x=\frac{b}{a}, y=2b$
    \task $x=\frac{a+2b}{7}, y=\frac{a-5b}{7}$
    \task $x=-\frac{26a+7b}{a^2-b^2}, y=\frac{7a+26b}{a^2-b^2}$
\end{tasks}
}

\clearpage
\chapter{Linear Relations in 3D.}

When a function or relation can be described using two variables, everything about the function can displayed in two dimensions.  The variable $x$ is used to represent one axis (usually left and right) and the other variable $y$ is used to represent the perpendicular axis (usually up and down).
\\[1em]
When we use three variables, we can represent objects in three dimensions.  The $z$-axis is generally thought to come out of our page.  So if you were to stick your pencil at the origin so it made a 90 degree angle to both the $x$- and $y$- axis, then that pencil would be the $z$-axis.
\\[1em]
A 3 dimensional axis along with the point $A(-3,2,1)$ is graphed below.\\[1em]
\begin{tikzpicture}
\begin{axis}[
  ticks=none,
  view={15}{30},
  axis lines=center,
  width=10cm,height=10cm,
  xmin=-5,xmax=5,ymin=-5,ymax=5,zmin=-5,zmax=5,
  xlabel={$x$},ylabel={$y$},zlabel={$z$}
]

% plot dots for the points
\addplot3 [only marks] coordinates {(-3,2,1)};

% plot dashed lines to axes
\addplot3 [no marks,densely dashed] coordinates {(-3,2,1)};

% label points
\node [above] at (axis cs:-3,2,1) {$A (-3,2,1)$};
\end{axis}
\end{tikzpicture}
\\[1em]
You may notice that it is difficult to visualize a 3 dimensional object on a 2 dimensional peice of paper.  This is where technology may be helpful.

\begin{figure}[h]
\begin{minipage}{.5\textwidth}
  \centering
	\caption*{Geogebra 3D for Android}
	\includegraphics[width=2in]{geogebra-android-qr}
\end{minipage}%
\begin{minipage}{.5\textwidth}
  \centering
	\caption*{Geogebra 3D for iOS}
	\includegraphics[width=2in]{geogebra-ios-qr}
\end{minipage}
\end{figure}

What does the linear relation $2x+3y+4z=0$ look like?
\\[4em]
What do the coefficients do to the relation?
\\[4em]
What do you think happens when two planes intersect?
\\[4em]
Graph the two planes:
\begin{align*}
		2x+3y-2z&=1\\
		-3x+2y+5z&=2.
\end{align*}
What do you notice about their intersection?
\\[4em]
How many equations would you need for their intersection to be a single point?
\\[4em]
A linear equation in three variables describes a plane, and all points $(x,y,z)$ that are \emph{coplanar} to that plane are solutions to the equation.
\prb{Find a solution to the equation $-x+2y-8z=0$.\\[6em]}
\sol{(Infinitely) many answers to this question.  One solution is $(0,0,0)$.  Extension: if $x=1$, and $y=2$, what is $z$?}
\prb{
Find two parallel planes. \\[6em]
}
\sol{(Infinitely) many answers to this question.  One such solution could be $x+y+z=0$ and $3x+3y+3z=1.$}
\prb{Describe the necessary and sufficient conditions for three planes to intersect at infinitely many points.\\[6em]}
\sol{This can happen in two ways:
\begin{tasks}
	\task all three planes are identical, or
	\task all three planes pass through the same line.
\end{tasks}
}
\prb{Describe the necessary and sufficient conditions for three planes that intersect nowhere.  (In other words, no point is on any two planes at the same time.)
\\[6em]
}
\sol{All three planes are parallel to one another, like a layered cake.}
\prb{
Describe the necessary and sufficient conditions for three planes to intersect, but not all three at the same place.  (In other words, there is no point that is on all three planes at the same time, but any two of the three do in fact intersect.)\\[6em]
}
\sol{If each plane intersects the other two with two parallel lines.  Roughly speaking, this would make a triangular tunnel.  For example,
	\begin{align*}
		x+0y+0z&=0\\
		0x+y+0z&=0\\
		x+y+0z&=1.
	\end{align*}
}
\prb{
Describe the necessary and sufficient conditions for three planes to intersect at exactly one point.
\\[8em]
}
\sol{Necessary: None of the planes can be parallel to any other plane. \\
Sufficient: There can only be one solution to x, one solution to y, and one solution to z.}
\prb{
What are the $x$-, $y$-, and $z$- intercepts of the equation $-4x+3y-6z=12$?\\[4em]
}
\sol{$x=-3$, $y=4$, $z=-2$.}
\prb{Are the two planes parallel? (You may use graphing software like Geogebra.)
\begin{align*}
	6x+2y-8z&=14\\
	-3x-y+4z&=3
\end{align*}
\vspace{1in}
}
\sol{Yes.}
\prb{How many solutions do the following systems have?  If the system has exactly one solution, state that solution.
\begin{tasks}(2)
	\task \vspace{-2.4em}
	\begin{align*}
		6x+2y-8z&=14\\
		-3x-y+4z&=3
	\end{align*}
	\task \vspace{-2.4em}
	\begin{align*}
		y-z&=0\\
		x-3z&=-1\\
		-x+3y&=1
	\end{align*}
	\task \vspace{-2.4em}
	\begin{align*}
		x-2y+3z&=9\\
		-x+3y &= -4\\
		2x-5y+5z&=17
	\end{align*}
	\task \vspace{-2.4em}
	\begin{align*}
		x-3y+z&=1\\
		2x-y-2z &=2\\
		x+2y-3z&=-1
	\end{align*}
	\end{tasks}
}
\sol{
\begin{tasks}(2)
	\task No solutions. (Inconsistent.)
	\task Infinitely many solutions. (Consistent.)
	\task One solution. (Consistent.)
	\task No solution. (Inconsistent.)
\end{tasks}\clearpage
}
\chapter{Solving Systems With Three Variables.}
In the last section we saw that a system of linear equations in three variables was equivalent to three planes in space.  To solve such a system is to find where (if anywhere) those three planes meet.  It turns out that the algebraic methods we used to solve systems of linear equations in two variables also work when we have three variables.  The only difference is that these are more work, and require some clever tricks.
\\[1em]
In particular, elimination can be used to eliminate one variable at a time until we have a single equation with one variable.  Before we practice using elimination on three variables, let's try an easy one.  Solve:
\begin{align*}
	x-2y+3z&=9\\
	y+3z&=5\\
	z&=2.
\end{align*}
\\[2in]
The method we used to solve this system is called back-substitution.  The system was easy to solve because it was in \emph{row-Echelon form}.  A system is said to be in row-Echelon form if the following properties are true:
\begin{itemize}
	\item Any equation $0=0$ is the bottom-most equation.
	\item The leading coefficient of each equation is 1 (called the leading 1).
	\item If an equation has a leading 1 on a variable, each equation below it has a coefficient of 0 on that variable.
\end{itemize}
The properties above will result in a staircase-shaped system.  Try writing a few systems in row-Echelon form:
\vspace{2in}
\subsection*{Gaussian Elimination}
Elimination with several variables requires some strategy.  One such strategy is named after mathematician Carl Friedrich Gauss.  In order to understand Gaussian Elimination, we first must understand the notion of an equivalent system.
\\[1em]
Two systems are said to be equivalent to one another if they share the same solutions.  One equivalent system may be obtained from another using one of three operations, called \emph{elementary row operations}.  The operations are as follows:
\begin{enumerate}
	\item one may interchange two equations,
	\item one may multiply one equation by a non-zero constant,
	\item one may add a multiple of one equation to another equation.
\end{enumerate}
The goal of using elementary row operations is to find an equivalent system that is in row-Echelon form.  While numerous algorithms exist, the best way to get good at Gaussian elimination is practice.  Some general tips:
\begin{itemize}
	\item If any equations start with zero, bring them ``down''.  The more zeros an equation begins with, the further it goes down.
	\item If any equation starts with 1, take advantage of that 1: clear out any coeficients below it, using the third elementary operation.
	\item Avoid using fractions where possible.
\end{itemize}
Solve the system
\begin{align*}
	x-2y+3z&=9\\
	-x+3y &=-4\\
	2x-5y+5z&=17
\end{align*}
\vfill
When a system has a single solution, Gaussian elimination will leave us with a system for which back-substitution can be employed: the last equation tells us exactly what the last variable is, and we work backwards to find the remaining variables.  What about systems that don't work out so nicely?
\clearpage
\subsection*{Inconsistent Systems}
Consider the following system.
\begin{align*}
x-3y+z&=1\\
2x-y-2z&=2\\
x+2y-3z&=-1
\end{align*}
\vfill
When a system ends up with 0 on one side of an equation and a non-zero constant on the other side of the equation, the system holds a contradiction.  That means that there are no possible values of each variable that can satisfy all of the equations.  This means we have no solutions to the system.  Another way of thinking about this is that the three planes do not intersect.  Either two or more are parallel, or they make a triangular tube.  In other words the system is inconsistent.
\clearpage
\subsection*{Parametric Solutions}
So far we have seen how to work with systems that have one solution and systems that have no solutions.  Don't forget that some systems have infinitely many solutions!  This is how we deal with those.
\\[1em]
Consider the following system:
\begin{align*}
y-z &= 0 \\
x-3z &= -1 \\
-x+3y&=1
\end{align*}
\vfill
Adding the parameter $t$ allows us to vary our solution.  One way to think about $t$ is as a slider in Desmos or Geogebra.  Try putting this system into Geogebra, and adding the point we found, along with an appropriate slider.
\clearpage
\prb{Solve each system using Gaussian Elimination.
\begin{tasks}(2)
	\task \vspace{-2.4em}
		\begin{align*}
		x-y&=4\\
		2y+z&=6\\
		3z&=6
		\end{align*}
	\task \vspace{-2.4em}
		\begin{align*}
		\begin{array}{r}
		3 x_{1}-2 x_{2}+4 x_{3}=1 \\
		x_{1}+x_{2}-2 x_{3}=3 \\
		2 x_{1}-3 x_{2}+6 x_{3}=8
		\end{array}
		\end{align*}
		\vspace{4in}
	\task \vspace{-2.4em}
		\begin{align*}
		x+y+z&=6\\
		2x-y+z&=3\\
		3x-z&=0
		\end{align*}
	\task \vspace{-2.4em}
		\begin{align*}
		\begin{array}{r}
		5 x_{1}-3 x_{2}+2 x_{3}=3 \\
		2 x_{1}+4 x_{2}-x_{3}=7 \\
		x_{1}-11 x_{2}+4 x_{3}=3
		\end{array}
		\end{align*}
		\vspace{4in}
		\task \vspace{-2.4em}
			\begin{align*}
			\begin{array}{rrrr}
				2 x_{1}&+x_{2} & -3x_{3}&=4 \\
				4 x_{1}&       & +2x_{3}&=10 \\
				-2x_{1}&+3x_{2}&-13x_{3}&=-8
			\end{array}
			\end{align*}
			\task \vspace{-2.4em}
				\begin{align*}
				\begin{array}{rrrr}
					3 x_{1}&+ 6x_{2} & -6x_{3}&=9 \\
					2 x_{1}&- 5x_{2} & +4x_{3}&=6 \\
					-1x_{1}&+16x_{2} &-14x_{3}&=-3
				\end{array}
				\end{align*}
\end{tasks}
}
\sol{
\begin{tasks}(2)
	\task $x=6, y=2, z=2)$
	\task No solution.
	\task $x=1, y=2, z=3)$
	\task No solution.
	\task $x_2=19-8x_1$, $x_3=5-2x_1$.
	\task $x_2=4x_1-12$, $x_3=\frac{9x_1-27}{2}$.
\end{tasks}
}
\clearpage
\prb{Is it possible for a system of 3 linear equations with 3 variables to have exactly one solution?}
\sol{Yes.}
\vspace{2in}
\prb{Is it possible for a system of 2 linear equations with 3 variables to have one solution?  Why?}
\sol{No.  Either they are parallel (inconsistent) or they intersect along a line (infinitely many solutions) or they are the same plane (infinitely many solutions).}
\vspace{2in}
\prb{Suppose I have a system of $m$ linear equations in $n$ variables.  What are the necessary conditions for me to have exactly one solution?}
\sol{$m\ge n$.}

\chapter{Matrices}
In the last section we learned how to solve a system of linear equations with many variables.  You may have noticed that it was tedious writing $x$, $y$, and $z$ at each stage of operation.  Luckily, there is a more convenient way of solving these systems, which is to use what is called a matrix (plural: matrices).
\\[1em]
An $m\times n$ matrix is:
\\[2in]
An augmented matrix is:
\\[2in]
Write the system as an augmented matrix:
\begin{align*}
\begin{array}{rrrrr}
	2 x_{1}&- 8x_{2} & -6x_{3}&=&1 \\
	1 x_{1}&+ 2x_{2} & +5x_{3}&=&-2 \\
	-3x_{1}&+ 1x_{2} & +2x_{3}&=&8
\end{array}
\end{align*}
\clearpage
\subsection*{Elementary row operations.}
All elementry row operations can be applied to a matrix.  You can think of a matrix as just a short-hand way of writing a system of linear equations.  \\[1em]
Since systems of linear equations sometimes take several steps to solve, it is important to write each operation you use along the way.
\\[1em]
Use a matrix to solve this system:

\begin{align*}
\begin{array}{rrrrr}
	2 x&- 2y & +3z&=&9 \\
	- x&+ 3y &    &=&-3 \\
	2 x&- 5y & +5z&=&17
\end{array}
\end{align*}
\clearpage
\subsection*{Solving a $4\times 4$ system.}
Some systems can have many variables and many equations.  Try solving this one:
\begin{align*}
\begin{array}{rrrrrr}
	       &   x_{2} &   x_{3}-2x_{4}&=&-3 \\
	  x_{1}&+ 2x_{2} & - x_{3}       &=& 2 \\
	 2x_{1}&+ 4x_{2} &   x_{3}-3x_{4}&=&-2 \\
	  x_{1}&- 4x_{2} &-7 x_{3}- x_{4}&=&-19
\end{array}
\end{align*}
\clearpage
\subsection*{A system with no solutions.}
Remember that some systems have no solutions.  How would this look in a matrix? Solve this one.
\begin{align*}
\begin{array}{rrrrr}
	  x&- 1y & +2z&=&4 \\
	  x&     &  +z&=&6 \\
	2 x&- 3y & +5z&=&4 \\
	3 x&  2y & -1z&=&1
\end{array}
\end{align*}
\clearpage
\subsection*{Infinitely many solutions.}
Solve this system by row-reducing an augmented matrix.
\begin{align*}
\begin{array}{rrrrr}
	  2x&+4y & -2z&=&0 \\
	  3x& +5y    &  &=&1 \\
\end{array}
\end{align*}
\clearpage
\prb{Solve the system that corresponds to the given augmented matrix.  Assume the variables are $x_1,x_2,\ldots$ and so on.
\begin{tasks}(2)
	\task \vspace{-2.3em}
		\begin{align*}
		\begin{amatrix}{2}
		1&0&0\\
		0&1&2
		\end{amatrix}
		\end{align*}
	\task \vspace{-2.3em}
		\begin{align*}
		\begin{amatrix}{3}
		1&-1&0&3\\
		0&1&-2&1\\
		0&0&1&-1
		\end{amatrix}
		\end{align*}
		\vspace{2in}
	\task \vspace{-2.3em}
		\begin{align*}
		\begin{amatrix}{3}
		2&1&-1&3\\
		1&-1&1&0\\
		0&1&2&1
		\end{amatrix}
		\end{align*}
	\task \vspace{-2.3em}
		\begin{align*}
		\begin{amatrix}{4}
		1&2&0&1&4\\
		0&1&2&1&3\\
		0&0&1&2&1\\
		0&0&0&1&4
		\end{amatrix}
		\end{align*}
\end{tasks}
}
\sol{
\begin{tasks}(2)
	\task $x_1=0,x_2=2$.
	\task $x_1=-1,x_2=-1,x_3=2$.
	\task $x_1=1,x_2=1,x_3=0$.
	\task $x_1=-26,x_2=13,x_3=-7,x_4=4$
\end{tasks}
}
\clearpage
\prb{Solve each system using Gaussian elimination and back-substitution.
\begin{tasks}(2)
	\task \vspace{-2.3em}
		\begin{align*}
		\begin{array}{rrrrr}
			  x &+&2y&=&7 \\
			  2x&+& y&=&8 \\
		\end{array}
		\end{align*}
	\task \vspace{-2.3em}
		\begin{align*}
		\begin{array}{rrrrr}
			  -x&+&2y&=&1.5 \\
			  2x&-&4y&=&3 \\
		\end{array}
		\end{align*}
		\vspace{4in}
	\task \vspace{-2.3em}
		\begin{align*}
		\begin{array}{rrrrr}
			  -3x&+&5y&=&-22 \\
			   3x&+&4y&=&4 \\
			  4x &-&8y&=&32
		\end{array}
		\end{align*}
	\task \vspace{-2.3em}
		\begin{align*}
		\begin{array}{rrrrrrr}
			  x &&&   -&3z&=&-2 \\
			  3x&+& y&-&2z&=&5 \\
			  2x&+&2y&+& z&=&4 \\
		\end{array}
		\end{align*}
		\vspace{4in}
	\task \vspace{-2.3em}
	\begin{align*}
		\begin{array}{rrrrrrr}
			x&+&  &-&5z&=&3 \\
			x&+& y&-&2z&=&1 \\
			2x&-&2y&-& z&=&0
		\end{array}
	\end{align*}
	\vspace{5in}
	\task \vspace{-2.3em}
		\begin{align*}
		\begin{array}{rrrrrrrrr}
			4w&+&12x &-&7y&-&20z&=&22 \\
			3w&+&9x  &-&5y&-&28z&=&30
		\end{array}
		\end{align*}
		\clearpage
		\task \vspace{-2.3em}
		\begin{align*}
			\begin{array}{rrrrrrr}
			 3x&+&3y&+&12z&=&6 \\
				x&+& y&+& 4z&=&2 \\
			 2x&+&5y&+&20z&=&10 \\
			 -x&+&2y&+& 8z&=&4
			\end{array}
		\end{align*}
		\task \vspace{-2.3em}
		\begin{align*}
			\begin{array}{rrrrrrrrr}
				2w&+& x &-& y&+&2z&=&-6 \\
				3w&+&4x & &  &+& z&=&1  \\
				 w&+&5x &+&2y&+&6z&=&-3 \\
				5w&+&2x &-& y&-& z&=&3
			\end{array}
		\end{align*}
\end{tasks}
}
\clearpage
\
\newpage
\sol{
\begin{tasks}(2)
	\task $x=3,y=2$.
	\task No solution.
	\task $x=4,y=-2$.
	\task $x=4,y=-3,z=2$.
	\task $x=-1/3$, $y=0$, $z=-2/3$.
	\task $y=\frac{13w+39x-4}{24}$, $z=\frac{w+3x-100}{96}$.
	\task $x=0$, $y=2-4z$
	\task $w=1,x=0,y=4,z=-2$.
\end{tasks}
 }
\chapter{Selected Solutions.}
\shipoutAnswer
\clearpage
Works cited:
\\[1em]
Mickelson, R. J., and Paul Mickelson. \emph{Theory and Problems for Grade 10 Math: Foundations of Mathematics and Pre-Calculus.} Crescent Beach Pub., 2009.
\\[1em]
Larson, Ron, and David C. Falvo. \emph{Elementary Linear Algebra.} Houghton Mifflin Harcourt Pub. Co., 2009.
\end{document}
