\documentclass[12pt]{article}
\usepackage{amsmath, amssymb}
\usepackage{geometry}
\usepackage{fancyhdr}
\usepackage{enumitem}
\usepackage{multicol}
\usepackage[most]{tcolorbox}

% Page setup
\geometry{margin=1in}
\setcounter{section}{1} % ensures numbering starts at 2.x

% Fancy headers
\pagestyle{fancy}
\fancyhf{}
\lhead{Grade 10 Advanced Math}
\rhead{Unit 2: Radicals}
\cfoot{\thepage}

% Note box for definitions/examples
\newtcolorbox{notebox}[1][]{
  colback=white!97!gray,
  colframe=black,
  fonttitle=\bfseries,
  coltitle=black,
  enhanced,
  sharp corners,
  boxrule=0.8pt,
  title=#1
}

\begin{document}

\section{Square Roots and Cube Roots}
\begin{notebox}[Definition]
A \textbf{square root} of $a$ is a number $b$ such that $b^2 = a$.  
A \textbf{cube root} of $a$ is a number $b$ such that $b^3 = a$.
\end{notebox}

\noindent Examples:
\[
\sqrt{25} = \_\_\_\_\,, \qquad \sqrt[3]{-27} = \_\_\_\_ 
\]

\begin{notebox}[Practice]
Evaluate:
\[
\sqrt{49}, \quad \sqrt[3]{125}, \quad \sqrt[3]{-8}
\]
\end{notebox}

\section{Simplifying Radicals}
\begin{notebox}[Key Idea]
Factor the radicand into perfect squares/cubes.  
Example: $\sqrt{50} = \_\_\_\_\_$
\end{notebox}

\begin{notebox}[Practice]
Simplify:
\[
\sqrt{72}, \quad \sqrt[3]{54}, \quad \sqrt{18x^2}
\]
\end{notebox}

\section{Exponential Notation}
\begin{notebox}[Definition]
Radicals can be written as fractional exponents:
\[
\sqrt[n]{a} = a^{1/n}
\]
\end{notebox}

\noindent Examples:
\[
\sqrt{a} = a^{1/2}, \qquad \sqrt[3]{a^2} = a^{2/3}
\]

\begin{notebox}[Exponent Laws]
\[
a^m \cdot a^n = a^{m+n}, \quad (a^m)^n = a^{mn}
\]
\end{notebox}

\begin{notebox}[Practice]
Write in exponential form:
\[
\sqrt[4]{x^3}, \quad \sqrt{y^5}
\]
\end{notebox}

\section{Rational vs Irrational Numbers}
\begin{notebox}[Definitions]
\textbf{Rational:} can be written as $\tfrac{p}{q}$ with $q \neq 0$.  
\textbf{Irrational:} cannot be expressed as a fraction.
\end{notebox}

\begin{notebox}[Example]
Proof that $\sqrt{2}$ is irrational:  
(Students complete outline of contradiction proof.)
\end{notebox}

\begin{notebox}[Practice]
Classify:
\[
\sqrt{9}, \quad \sqrt{7}, \quad 0.333\ldots, \quad \pi
\]
\end{notebox}

\section{Imaginary Numbers}
\begin{notebox}[Definition]
Define $i = \sqrt{-1}$.  
Then $i^2 = -1$.
\end{notebox}

\noindent Examples:
\[
\sqrt{-9} = \_\_\_\_\,, \qquad i^3 = \_\_\_\_, \qquad i^4 = \_\_\_\_
\]

\begin{notebox}[Practice]
Simplify:
\[
\sqrt{-25}, \quad i^3, \quad i^4
\]
\end{notebox}

\section{Number Systems}
\begin{notebox}[Hierarchy]
\begin{multicols}{2}
\begin{itemize}[topsep=0pt]
  \item $\mathbb{N}$: natural numbers
  \item $\mathbb{Z}$: integers
  \item $\mathbb{Q}$: rationals
  \item Irrationals
  \item $\mathbb{R}$: reals
  \item $\mathbb{C}$: complex numbers
\end{itemize}
\end{multicols}
\end{notebox}

\vspace{1cm}
\noindent\textbf{Diagram:} (Students draw a Venn diagram of number systems.)

\end{document}
