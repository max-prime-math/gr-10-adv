\setlength\parindent{0pt}

\usepackage{amsmath,amssymb,amsfonts,graphicx,tasks,tikz,pgfplots}
\usetikzlibrary{arrows,quotes}

\usepackage{array}
\newcommand{\PreserveBackslash}[1]{\let\temp=\\#1\let\\=\temp}
\newcolumntype{C}[1]{>{\PreserveBackslash\centering}p{#1}}
\newcolumntype{R}[1]{>{\PreserveBackslash\raggedleft}p{#1}}
\newcolumntype{L}[1]{>{\PreserveBackslash\raggedright}p{#1}}

\newcommand{\graph}[8]{
\begin{tikzpicture}[baseline=(current bounding box.north)]
  \begin{axis}[
    axis lines=middle,
    axis line style={very thick},
    grid style={thin,densely dotted,black!50},
    grid=major,
      xtick distance=#6, ytick distance=#6,
      xmin=#2-.2, xmax=#3+.2,
      ymin=#4-.2, ymax=#5+.2,
      xlabel=$x$,
      ylabel=$y$,
      #7]
      \addplot[
      domain=#2:#3,
      samples=500,#8]
      (x,{#1});
    \end{axis}
  \end{tikzpicture}
}

\newcommand{\graphtwo}[9]{
\begin{tikzpicture}[baseline=(current bounding box.north)]
  \begin{axis}[
    axis lines=middle,
    axis line style={very thick},
    grid style={thin,densely dotted,black!50},
    grid=major,
      xtick distance=#7, ytick distance=#7,
      xmin=#3-.2, xmax=#4+.2,
      ymin=#5-.2, ymax=#6+.2,
      xlabel=$x$,
      ylabel=$y$,
      #8]
      \addplot[
      domain=#3:#4,
      samples=500,#9]
      (x,{#1});
      \addplot[
      domain=#3:#4,
      samples=500,#9]
      (x,{#2});
    \end{axis}
  \end{tikzpicture}
}

\newcommand{\curve}[7]{% function, xmin, xmax, ymin, ymax, 
\begin{tikzpicture}[baseline=(current bounding box.north)]
  \begin{axis}[
    axis lines=middle,
    axis line style={very thick},
    grid style={thin,densely dotted,black!50},
    grid=major,
    xticklabels={},
	yticklabels={},
	xtick distance=1, ytick distance=1,
	extra x ticks={#3},
	extra y ticks={#5},
      xmin=#2-.2, xmax=#3+.2,
      ymin=#4-.2, ymax=#5+.2,
      xlabel=$x$,
      ylabel=$y$,
      #6]
      \addplot[
      domain=#2:#3,
      samples=500,#7]
      (x,{#1});
    \end{axis}
  \end{tikzpicture}
}

\newcommand{\blankgraph}[6]{
\begin{tikzpicture}[baseline=(current bounding box.north)]
  \begin{axis}[
    width=#5in,height=#6in,
    axis lines=middle,
    axis line style={very thick},
    grid style={thin,densely dotted,black!50},
    grid=major,
    xticklabels={},
	yticklabels={},
	xtick distance=1, ytick distance=1,
	extra x ticks={#2},
	extra y ticks={#4},
      xtick distance=1, ytick distance=1,
      xticklabels={1},
      yticklabels={1},
      xmin=#1-.2, xmax=#2+.2,
      ymin=#3-.2, ymax=#4+.2,
      xlabel=$x$,
      ylabel=$y$]
    \end{axis}
  \end{tikzpicture}
}

\newcommand{\blankaxes}[2]{
\begin{tikzpicture}[baseline=(current bounding box.north)]
  \begin{axis}[
    width=#1in,height=#2in,
    axis lines=middle,
    axis line style={very thick},
    grid=major,
    xtick distance=2, ytick distance=2,
    xmin=-1, xmax=1,
    ymin=-1, ymax=1,
    xlabel=$x$,
    ylabel=$y$]
  \end{axis}
\end{tikzpicture}
}

\newcommand{\ds}{\displaystyle}

\newcommand{\lr}[1]{\left(#1\right)}

% \usepackage{amsthm}
% \theoremstyle{definition}
% \newtheorem{problem}{}
% \counterwithin*{problem}{section}
% \renewcommand{\theproblem}{\thesubsection.\arabic{problem}}
%
% \newtheorem*{solutionx}{\thesolutionnumber}
% \ExplSyntaxOn
% \tl_new:N \g_gargantuar_solution_tl
%
% \NewDocumentEnvironment{solution}{+b}
%  {
%   \tl_gput_right:Nx \g_gargantuar_solution_tl
%    {
%     \printsolution{\theproblem}{ \exp_not:n { #1 } }
%    }
%  }
%  {}
% \NewDocumentCommand{\printsolutions}{}
%  {
%   \tl_use:N \g_gargantuar_solution_tl
%   \tl_gclear:N \g_gargantuar_solution_tl
%  }
% \NewDocumentCommand{\printsolution}{m +m}
%  {
%   \cs_set:Npn \thesolutionnumber { #1 }
%   \begin{solutionx} #2 \end{solutionx}
%  }
% \ExplSyntaxOff
% \makeatother
%
% \newcommand{\prb}[1]{\begin{problem}#1\end{problem}}
% \newcommand{\sol}[1]{\begin{solution}#1\end{solution}}
